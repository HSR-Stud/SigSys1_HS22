
\section{Signalflussdiagramme}
\subsection{Begriffe}
\begin{tabular}{|p{0.3\textwidth}|p{0.6\textwidth}|}
\hline
\textbf{Knoten:} &
Darstellung einer Grösse, eines Signals oder einer Variable
\\
\hline
\textbf{Zweig:} &
Funktionelle Abhängigkeit einer Grösse
\\
\hline
\textbf{Quelle:} &
Unabhängiger Knoten, es münden keine Zweige ein  
\\
\hline
\textbf{Senke:} &
Knoten ohne weggehende Zweige
\\
\hline
\textbf{Pfad:} &
Kontinuierliche Folge von Zweigen, die in die gleiche Richtung zeigen
\\
\hline
\textbf{Offener Pfade:} &
Ein Pfad, bei dem jeder beteiligte Knoten nur einmal durchquert wird
\\
\hline
\textbf{Vorwärtspfad:} &
Ein offener Pfad zwischen einer Quelle und einer Senke
\\
\hline
\textbf{Schleife (L):} &
Ein geschlossener Pfad, welcher zum Ausgangsknoten zuruckkehrt, wobei jeder beteiligte
Knoten nur einmal durchlaufen wird, ausgenommen der Ausgangsknoten
\\
\hline
\textbf{Eigenschleife:} & 
Eine (Rückkopplungs)schleife, die aus einem Zweig und einem Knoten besteht 
\\
\hline
\textbf{Zweigtransmittanz:} &
Die lineare Grösse, unabhängig von ihrer Dimension, die einen Knoten eines Zweiges zum
anderen Knoten in Beziehung setzt.
\\
\hline
\textbf{Schleifentransmittanz:} &
Das Produkt der Zweigtransmittanzen in einer Schleife.
\\
\hline
\end{tabular}
