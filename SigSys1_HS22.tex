% !TeX program = xelatex
% !TeX encoding = utf8
% !TeX root = SigSys1_HS22.tex

%% TODO: publish to CTAN
\documentclass[margin=normal]{tex/hsrzf}

%%%%%%%%%%%%%%%%%%%%%%%%%%%%%%%%%%%%%%%%%%%%%%%%%%%
% Packages

%% TODO: publish to CTAN
\usepackage{tex/hsrstud}

%% Language configuration
\usepackage{polyglossia}
\setdefaultlanguage[variant=swiss]{german}

%% License configuration
\usepackage[
    type={CC},
    modifier={by-nc-sa},
    version={4.0},
    lang={german},
]{doclicense}

%other Packages
\usepackage{multicol,multirow}
\usepackage[export]{adjustbox}
%amssymb,amsmath,fancybox,graphicx,color,lastpage,
%wrapfig,fancyhdr,hyperref,verbatim,floatflt,
%multicol,multirow,rotating,pdflscape,array,longtable

%%%%%%%%%%%%%%%%%%%%%%%%%%%%%%%%%%%%%%%%%%%%%%%%%%%
% Metadata

\course{Elektrotechnik}
\module{SigSys}
\semester{Herbstsemester 2022}

\authoremail{joel.leirer@ost.ch}
\author{\textsl{Joël Leirer} -- \texttt{\theauthoremail}}

% did someone help you with this work?
\contributors{

}

\title{\texttt{\themodule} Zusammenfassung}
\date{\thesemester}

%%%%%%%%%%%%%%%%%%%%%%%%%%%%%%%%%%%%%%%%%%%%%%%%%%%
% Document

\begin{document}

% use roman numberals for introductiory pages
\pagenumbering{roman}

\maketitle

% \begin{abstract}
% \end{abstract}

% show the names of the people who contributed to this document.
% \section*{Contributors}
% \thecontributors

\section*{Lizenz}
\doclicenseThis
\clearpage
\tableofcontents

% actual content
\clearpage
\setcounter{page}{1}
\pagenumbering{arabic}

\input{include/Signalklassen/Signalklassen.tex}
\section{Kenngrössen von Signalen}
\begin{tabular}{p{6cm}p{12cm}}
  Linearer Mittelwert
  \newline \tiny(auch: $ \bar{x}, x_m$)                                         &
  $X_0 = \frac{1}{T} \int \limits _{-T/2}^{T/2} x(t) dt $                         \\
  Quadratischer Mittelwert
  \newline  \tiny(nur Signale Klasse 2a \newline
  Klasse 2b mit: $\lim_{t \to \infty}$)                                         &
  $X^2 = \frac{1}{T} \int \limits _{-T/2}^{T/2} x^2(t) dt$                        \\
  Effektivwert \newline \tiny{("Quadratischer Mittelwert", RMS)}                &
  $X^2 = \frac{1}{T} \int \limits _{-T/2}^{T/2} \sqrt{x^2(t)} dt $                \\
  Mittelwert n. Ordnung \newline \tiny(nur Signale $\in \mathbb{R}$, Klasse 2a) &
  $X^n = \frac{1}{T} \int \limits _{-T/2} ^{T/2} x^n(t)dt$                        \\
  Varianz                                                                       &
  $Var(x) = \sigma^2 = \frac{1}{T} \int \limits _{-T/2} ^{T/2} (x(t) - X_0)^2 dt$ \\
  Standardabweichung                                                            &
  $\sigma = \sqrt{Var(x)} = \sqrt{X^2 - (X_0)^2}$                                 \\
\end{tabular}

\subsubsection*{Die Autokorrelationsfunktion (AKF)}
$$ \varphi_{xx}(\pm \tau) = \frac{1}{T} \int \limits _{-T/2} ^{T/2} x(t) \cdot x(t- \tau) dt
  = \frac{1}{T} \int \limits _{-T/2} ^{T/2} x(t + \tau) \cdot x(t)dt$$
\subsubsection*{Eigenschaften:}
\begin{itemize}
  \item $\varphi_{xx}(0) = X^2 = (X_0)^2 + \sigma^2$ \tiny Quadratischer Mittelwert \normalsize
  \item $\varphi_{xx}(\tau) = \varphi_{xx}(\tau \pm n \cdot T)$, mit $n \in \mathbb{N}$,
        AKF hat gleiche Periode $T$ wie $x(t)$
  \item $\varphi_{xx}(\tau) = \varphi_{xx}(-\tau)$, AKF ist gerade Funktion
  \item $\varphi_{xx}(0) \geq \left|\varphi_{xx}(\tau)\right|$
  \item $\varphi_{xx}(\tau) \geq (X_0)^2 - \sigma^2$
\end{itemize}

\subsection{Vergleich Signalleistung / physikalische Leistung}
Leistungsverhältnisse zweier Leistungen wird oft in dB, Dezibel angegeben.
Bel steht für das Verhältnis zweier Werte im Zehnerlogarithmus.
Aufgrund des d (=dezi) muss ein Faktor 10 verwendet werden.
Werden anstelle Leistungen Effektivwerte genommen wird ein Faktor 20 benötigt.
$$10 \cdot \log_{10} (\frac{P_y}{P_x}) = 
10 \cdot \log_{10}(\frac{(y_{rms})^2}{(x_{rms})^2}) = 
20 \cdot log_{10}(\frac{y_{rms}}{x_{rms}}) = k[\textrm{dB}]$$ 
daraus folgt:
$$P_y = P_x \cdot 10^{\frac{k}{10}} \; P_x = \frac{P_y}{10^{\frac{k}{10}}}$$
bzw:
$$y_{rms} = x_{rms} \cdot 10^{\frac{k}{20}} \; x_{rms} = \frac{y_{rms}}{10^{\frac{k}{20}}}$$
%needs Packages:
% - \usepackage[export]{adjustbox} for  "valign=t"

\section{Wichtige Funktionen}
\begin{tabular}{p{5cm} p{12.5cm}}
  \includegraphics[width = 5cm, valign=t]{include/Wichtige Funktionen/img/Sprungfunktion.png} &
  Sprungfunktion (Heaviside):
  Normierter Einschaltvorgang
  $$H(t) = \begin{cases}
               0 \textrm{ für }  t<0,                                                                      \\
               [\frac{1}{2} \textrm{ für }  t = 0,] \textrm{ \tiny(nicht immer vorhanden; dann 1 für t=0)} \\
               1 \textrm{ für }  t >0.
             \end{cases}   $$
  \\
  \includegraphics[width = 5cm, valign=t]{include/Wichtige Funktionen/img/Impulsfunktion.png} &
  Diracimpuls \tiny (auch Impuls-/Deltafunktion, Delta-Distribution)
  \normalsize \newline
  Unendlicher kurzer normierter Impuls mit unendlicher Amplitude
  $$\int\limits _{-\infty} ^{+\infty} f(t) \cdot \delta (t-t_0) dt = f(t_0) \;
    \int\limits _{-\infty} ^{+\infty} f(t) \cdot \delta (t) dt = f(0) \;
  \int\limits _{-\infty} ^{+\infty} \delta (t) dt = 1$$                                 \\
  \includegraphics[width = 5cm, valign=t]{include/Wichtige Funktionen/img/Signumfunktion.png} &
  Signumfunktion (Vorzeichenfunktion)
  $$sgn(t) = \begin{cases}
                 -1 \textrm{ für }  t<0,  \\
                 0 \textrm{ für }  t = 0, \\
                 1 \textrm{ für }  t >0.
               \end{cases}   $$                                                   \\
  \includegraphics[width=5cm, valign=t]{include/Wichtige Funktionen/img/Rampenfunktion.png}   &
  Rampenfunktion 
  $$r(t) = \begin{cases}
               0 \textrm{ für } t \leq 0, \\
               t \textrm{ für } t > 0.
             \end{cases}$$\\
\end{tabular}

\input{include/Fourier Reihe/Fourier-Reihe.tex}


\section{Anhang}
\input{include/Wichtige Werte & Vereinfachungen/Wichtige Werte & Vereinfachungen.tex}
\includegraphics{img/Tabelle_Dichtefunktionen.png}

